\documentclass[main.tex]{subfiles}
 
\begin{document}

\chapterimage{band1.jpg}
\chapter{Connecting to Wi-Fi}

Let's now get this power outlet on a Wi-Fi network. In this Chapter we will connect to a hard-coded Wi-Fi network that is embedded within the device's firmware executable image. You may refer to the \textit{3\_wifi\_connection/} directory of esp-jumpstart for looking at this code.

Wi-Fi is a protocol that can generate asynchronous events like connectivity lost, connection established, DHCP Address received etc. For this, we register a handler with the Wi-Fi and network subsystem. This handler will get called whenever either of these asynchronous events occurs.

\section{The Code}\index{The Code}
\begin{minted}{c}
#include <esp_wifi.h>
#include <esp_event_loop.h>

tcpip_adapter_init();
esp_event_loop_init(event_handler, NULL);

wifi_init_config_t cfg = WIFI_INIT_CONFIG_DEFAULT();
esp_wifi_init(&cfg);
esp_wifi_set_mode(WIFI_MODE_STA);

wifi_config_t wifi_config = {
    .sta = {
        .ssid = EXAMPLE_ESP_WIFI_SSID,
        .password = EXAMPLE_ESP_WIFI_PASS,
    },
 };
esp_wifi_set_config(ESP_IF_WIFI_STA, &wifi_config);
esp_wifi_start();

\end{minted}

In the above code:
\begin{itemize}
    \item We initialize the TCP/IP stack with the \textit{tcpip\_adapter\_init()} call
    \item Similarly, the Wi-Fi subsystem and its station interface is initialized with the calls to \textit{esp\_wifi\_init()} and \textit{esp\_wifi\_set\_mode()}
    \item Finally, the hard-coded SSID and passphrase configuration of the target Wi-Fi network are configured and we start the station using a call to \textit{esp\_wifi\_start()}
\end{itemize}

The call to \textit{esp\_event\_loop\_init()} is important. The event loop collects events from the TCP/IP Stack and the Wi-Fi subsystem. It delivers these events to the callback that is registered through the first parameter.

\ksnotebox{The callback handler is executed from the event loop task. Care should be taken to ensure that this callback's execution doesn't overflow the event loop's stack. If you need a deeper stack in your callback, you can increase the event loop's stack size by changing the SDK configuration.}

The asynchronous event handler that is registered with the event loop can be implemented as:
\begin{minted}{c}
esp_err_t event_handler(void *ctx, system_event_t *event)
{
    switch(event->event_id) {
    case SYSTEM_EVENT_STA_START:
        esp_wifi_connect();
        break;
    case SYSTEM_EVENT_STA_GOT_IP:
        ESP_LOGI(TAG, "Connected with IP Address:%s",  
             ip4addr_ntoa(&event->event_info.got_ip.ip_info.ip));
        break;
    case SYSTEM_EVENT_STA_DISCONNECTED:
        esp_wifi_connect();
        break;
    return ESP_OK;
}
\end{minted}

The event handler current handles 3 events. When it receives an event \textit{SYSTEM\_EVENT\_STA\_START}, it asks the station interface to connect using the \textit{esp\_wifi\_connect()} call. The same action is taken even when we receive a Wi-Fi disconnect event.

The event \textit{SYSTEM\_EVENT\_STA\_GOT\_IP} is received when a DHCP IP address is obtained by ESP32. In this particular case, we  only print the IP address on the console.

\section{Progress so far}\index{Progress so far}
You can now modify the application to enter your Wi-Fi network's SSID and the passphrase. When you compile and flash this code on your development board, the ESP32 should connect to your Wi-Fi network and print the IP address on the console. The outlet's functionality of toggling the GPIO on pressing the push-button is, of course, also retained.

One problem with this approach is that the Wi-Fi settings are hard-coded into the firmware image. While this is ok for a hobby project, a product will require the end-user to dynamically configure this device with their settings. This is what we will look at in the next chapter.

\end{document}
