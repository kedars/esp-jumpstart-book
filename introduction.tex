\documentclass[main.tex]{subfiles}
 
\begin{document}
\chapterimage{band1.jpg}
\chapter{Introduction}

\section{ESP-Jumpstart: Build ESP32 Products Fast}\index{ESP32: Build ESP32 Products Fast}

Building production-ready firmware can be hard. It involves multiple questions and decisions about the best ways of doing things. It involves building phone applications, and integrating cloud agents to get all the features done. What if there was a ready reference, a known set of best steps, gathered from previous experience of others that you could jumpstart with?

ESP-Jumpstart is focused on building 'products' on ESP32. It is a quick-way to get started into your product development process. ESP-Jumpstart builds a fully functional, ready to deploy "Smart Power Outlet" in a sequence of incremental tutorial steps. Each step addresses either a user-workflow or a developer workflow. Each step is an application built with ESP-IDF, ESP32's software development framework.

Building your production firmware, is simply a matter of replacing the power-outlet's device driver, with your device driver (bulb, washing machine).


\begin{itemize}
\item An ESP32 development kit: ESP32-DevKit-C (Available on DigiKey, Mouser, Amazon)
\item A Development setup (\url{https://docs.espressif.com/projects/esp-idf/en/latest/get-started/})
\end{itemize}


\section{Hardware Overview}\index{Hardware Overview}
XXX

\end{document}
