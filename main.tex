%%%%%%%%%%%%%%%%%%%%%%%%%%%%%%%%%%%%%%%%%
%
% Important note:
% Chapter heading images should have a 2:1 width:height ratio,
% e.g. 920px width and 460px height.
%
%
% The original template (the Legrand Orange Book Template) can be found here --> http://www.latextemplates.com/template/the-legrand-orange-book
%
% Original author of the Legrand Orange Book Template:
% Mathias Legrand (legrand.mathias@gmail.com) with modifications by:
% Vel (vel@latextemplates.com)
%
% Original License:
% CC BY-NC-SA 3.0 (http://creativecommons.org/licenses/by-nc-sa/3.0/)
%%%%%%%%%%%%%%%%%%%%%%%%%%%%%%%%%%%%%%%%%
 
%----------------------------------------------------------------------------------------
%	PACKAGES AND OTHER DOCUMENT CONFIGURATIONS
%----------------------------------------------------------------------------------------

\documentclass[11pt,fleqn]{book} % Default font size and left-justified equations

\usepackage[top=3cm,bottom=3cm,left=3.2cm,right=3.2cm,headsep=10pt,letterpaper]{geometry} % Page margins

\usepackage[table]{xcolor} % Required for specifying colors by name
\definecolor{ocre}{RGB}{52,177,201} % Define the orange color used for highlighting throughout the book
\usepackage{tabularx}

% Font Settings
\usepackage{avant} % Use the Avantgarde font for headings
%\usepackage{times} % Use the Times font for headings
\usepackage{mathptmx} % Use the Adobe Times Roman as the default text font together with math symbols from the Sym­bol, Chancery and Com­puter Modern fonts

\usepackage{microtype} % Slightly tweak font spacing for aesthetics
\usepackage[utf8]{inputenc} % Required for including letters with accents
\usepackage[T1]{fontenc} % Use 8-bit encoding that has 256 glyphs
\usepackage{minted} % For code highlighting

\usepackage{marginnote}
% Bibliography
\usepackage[style=alphabetic,sorting=nyt,sortcites=true,autopunct=true,babel=hyphen,hyperref=true,abbreviate=false,backref=true,backend=biber]{biblatex}
\addbibresource{bibliography.bib} % BibTeX bibliography file
\defbibheading{bibempty}{}

\input{structure} % Insert the commands.tex file which contains the majority of the structure behind the template
\setlength{\parindent}{0pt} % Don't indent paragraphs
\setlength{\parskip}{1em}

\newcommand{\ksnotebox}[1]{\begin{tabularx}{\textwidth}{ |c|X| }
\hline
\cellcolor{lightgray} \textbf{Note} & #1 \\
\hline
\end{tabularx}} % Define a template for the note box

\begin{document}
\title{Building Products with ESP32 fast}

%----------------------------------------------------------------------------------------
%	TITLE PAGE
%----------------------------------------------------------------------------------------

\begingroup
\thispagestyle{empty}
\AddToShipoutPicture*{\put(0,0){\includegraphics[scale=0.6]{esahubble}}} % Image background
\centering
\vspace*{5cm}
\par\normalfont\fontsize{35}{35}\sffamily\selectfont
\textbf{Building Products with ESP32 \textit{fast}}\\
{\LARGE Jumpstart: From concept to production}\par % Book title
\vspace*{1cm}
{\Huge John Galt}\par % Author name
\endgroup

%----------------------------------------------------------------------------------------
%	COPYRIGHT PAGE
%----------------------------------------------------------------------------------------

\newpage
~\vfill
\thispagestyle{empty}

%\noindent Copyright \copyright\ 2018 Kedar Sovani\\ % Copyright notice

\noindent \textsc{For Espressif Systems}

\noindent \textsc{github.com/mahavirj/esp-jumpstart} % URL

\noindent This work is the culmination of years of learning in the IoT product development across a wide team of engineers at Espressif. The content started off as official training material. Given the success of the trainings in getting products off the ground, the material has subsequently been converted to this format. Feel free to make copies and share. % License information

\noindent \textit{First release, 2018} % Printing/edition date

%----------------------------------------------------------------------------------------
%	TABLE OF CONTENTS
%----------------------------------------------------------------------------------------

\chapterimage{band1.jpg} % Table of contents heading image

\pagestyle{empty} % No headers

\tableofcontents % Print the table of contents itself

%\cleardoublepage % Forces the first chapter to start on an odd page so it's on the right

\pagestyle{fancy} % Print headers again

%----------------------------------------------------------------------------------------
%	CHAPTER 1
%----------------------------------------------------------------------------------------

\chapterimage{band1.jpg} % Chapter heading image

\chapter{Introduction}

\section{ESP-Jumpstart: Build ESP32 Products Fast}\index{ESP32: Build ESP32 Products Fast}

ESP-Jumpstart is focused on building 'products' on ESP32. It is a quick-way to get started into your product development process. ESP-Jumpstart builds a fully functional, ready to deploy "Smart Power Outlet" in a sequence of incremental steps. Each step addresses either a user-workflow or a developer workflow.

And best of all, you need not be an IoT or ESP expert for using ESP-Jumpstart. All you need is:

\begin{itemize}
\item An ESP32 development kit: ESP32-DevitKit-C (Available on DigiKey, Mouser, Amazon)
\item A Development setup (\url{https://docs.espressif.com/projects/esp-idf/en/latest/get-started/})
\end{itemize}

Once you have ESP-Jumpstart functional, adapting it is simply a matter of replacing the power-outlet's device driver, with your device driver (bulb, washing machine).

\section{Hardware Overview}\index{Hardware Overview}
XXX




%----------------------------------------------------------------------------------------
%	CHAPTER 2
%----------------------------------------------------------------------------------------
\chapterimage{band1.jpg}

\chapter{Getting Started}

In this chapter, our aim would be to get our development setup functional, and also to get an understanding for the development tools and repositories available around ESP32.

\section{Development Overview}\index{Development Overview}

The following diagram depicts the typical developer setup for development with ESP.
\begin{figure}[h]
    \centering
    \includegraphics[scale=0.3]{Pictures/dev_setup.png}
    \caption{Typical Developer Setup}
    \label{fig:dev_setup}
\end{figure}

The PC, or the Development Host can be any of Linux, Windows or Mac. The ESP32 based development board is connected to the Development Host over a USB cable. The Development Host has the ESP-IDF (Espressif's SDK), the compiler toolchain and the code for your project. The development host builds this code and generates the executable firmware image. The tools on the Development Host then download the generated firmware image on to the development board. As the firmware executes on the development board, the logs from the firmware can be monitored from the Development Host.

\section{Getting ESP-Jumpstart}\index{Getting ESP-Jumpstart}

Let's get started by cloning the ESP-Jumpstart git repositories \url{https://github.com/mahavirj/esp-jumpstart}. This repository contains the sequence of applications that we will use for this exercise. This repository also contains a stable release version of IDF as a git submodule. This ensures that we are working off of a stable release of IDF.

\begin{verbatim}
$ git clone --recursive https://github.com/mahavirj/esp-jumpstart
\end{verbatim}

This repository already contains a copy of the IDF, Espressif's IoT Development Framework. Let's define the IDF\_PATH variable to point to the correction location of IDF. This can be done by executing the following command in your console:

\begin{verbatim}
$ export IDF_PATH=/path/to/esp-jumpstart/esp-idf
\end{verbatim}

Now let's make sure your development host (Windows, Linux or Mac) has the required packages to build and monitor ESP32 based projects. What you specifically need to do is:
\begin{itemize}
    \item Setting the toolchain \url{https://docs.espressif.com/projects/esp-idf/en/latest/get-started/#setup-toolchain}
    \item Establish serial connection with ESP32 \url{https://docs.espressif.com/projects/esp-idf/en/latest/get-started/establish-serial-connection.html}
\end{itemize}

\section{ESP-IDF}\index{ESP-IDF}

ESP-IDF is Espressif's IoT Development Framework. 
\begin{itemize}
    \item ESP-IDF is a collection of libraries and header files that provides the core software components that are required to build any software projects on ESP32. 
    \item ESP-IDF also provides tools and utilities that are required for typical developer and production usecases, like build, flash, debug and measure.
\end{itemize}

The IDF has a component-based design. 

\begin{figure}[h]
    \centering
    \includegraphics[width=\textwidth]{Pictures/idf_comp.png}
    \caption{Component Based Design}
    \label{fig:idf_comp_design}
\end{figure}

All the software in the IDF is available as components. The Operating System, the network stack, Wi-Fi drivers, middleware modules like the HTTP Server are all components within IDF. 

This design allows you to use your own or third-party components that are built for ESP-IDF.

A developer typically builds \textit{applications} against the IDF. The applications typically contains the business logic, any drivers for externally interfaced peripherals and the SDK configuration.

\begin{figure}[h]
    \centering
    \includegraphics[scale=0.1]{Pictures/app_structure.png}
    \caption{Application's Structure}
    \label{fig:app_structure}
\end{figure}

An application must contain one \textit{main} component. This is the primary component that holds the application logic. The application may additionally include other components as may be desired.
The application's \textit{Makefile} defines the build instructions for the application. 
Additionally, an optional \textit{sdkconfig.defaults} may be placed that picks up the default SDK configuration that should be selected for this application. More details about the SDK configuration follow.

\subsection{SDK Configuration}\index{SDK Configuration}

Given that this is an embedded application with footprint constraints, the IDF allows every application to choose its own SDK configuration. The SDK configuration allows you to select specific configuration options of the SDK that suit your application.

Typically, there is a feature v/s footprint tradeoff, where pulling in a new feature will consume greater memory footprint.

Let us now first use the \textit{Hello World} application and launch the SDK configuration for this application.

\begin{verbatim}
$ cd esp-jumpstart/1hello_world
$ make -j8 menuconfig
\end{verbatim}

This will first open a pop-up screen for the SDK configuration.

For our current scenario, we will choose the default configuration. Later in this series, we will look at greater details into the SDK configuration options. For now, you can simply exit from this screen. On exiting, when asked for a prompt whether you want to save the SDK configuration, say "Yes".

\ksnotebox{If you are building the application for the first time, the SDK configuration screen will pop-up automatically even if you build for some other build target. For subsequent builds this SDK configuration screen wouldn't show up again unless you specify the \textit{menuconfig} target.}

\subsection{Build and Flash}\index{Build and Flash}
\begin{verbatim}
$ make -j8 flash monitor
\end{verbatim}

The SDK will then build the entire SDK and the application. Once the build is successful, it will write the generated firmware to the device.

At this point you should have connected the device to your development host. If you have installed the correct drivers, you should see a new device in your machine. For Windows, a new COM port would have been created. For Linux/OSX, a new file would appear in /dev/tty.*. The flashing utility should know which serial port is connected to your device. This can be configured by setting the 'ESPPORT' environment variable. The following command should help:
\begin{verbatim}
$ export ESPPORT=/dev/tty.SLAB_USBTOUART
\end{verbatim}

The rate at which the flashing utility writes firmware to the device can also be configured. This is called the baud rate. The typical rate is 115200. But this can be extended upto 921600. Let's configure the maximum baud rate for our flashing.
\begin{verbatim}
$ export ESPBAUD=921600
\end{verbatim}

\ksnotebox{On some development boards you may have to press a specific button configuration in order to put the development board into the 'flashing mode'. Please refer to your development board's documentation for these details. For ESP32-DevKit-C, no such button press is required, the board is automatically put into the flashing mode by the flasher utility.}

\section{The Code}\index{The Code}
Now let's look at the code of the Hello World Application. It is only a few lines of code as shown below:
\begin{minted}{c}
#include <stdio.h>
#include "freertos/FreeRTOS.h"
#include "freertos/task.h"


void app_main()
{
    int i = 0;
    while (1) {
        printf("[%d] Hello world!\n", i);
        i++;
        vTaskDelay(5000 / portTICK_PERIOD_MS);
    }
}
\end{minted}
The code is fairly simple. A few takeaways:
\begin{itemize}
\item The app\_main() function is the application entry point. All applications begin execution at this point. This function gets called after the FreeRTOS kernel is already executing on both the cores of the ESP32. Once FreeRTOS is initialis\
ed, it forks an application thread, called the main thread, on one of the cores. The app\_main() function is called in this thread's context. The stack of the application thread can be configured through the SDK configuration.
\item C library functions like printf(), strlen(), time() can be directly called. The IDF uses the newlib C library, which is a low-footprint implementation of the C library. Most of the category of functions of the C library like stdio, stdlib, string operations, math, time/timezones, file/directory operations are supported. Support for signals, locales, wchrs is not available. In our example above, we use the printf() function for printing to the console.
\item FreeRTOS is the operating system powering both the cores. FreeRTOS (https://www.freertos.org) is a tiny kernel that provides mechanisms for task creation, inter-task communication (sempahores, message queues, mutexes), interrupts and timers. In our example above, we use the vTaskDelay function for putting the thread to sleep for 5 seconds. Details of the FreeRTOS APIs are available at: https://www.freertos.org/a00106.html
\end{itemize}

\section{Progress so far}\index{Progress so far}
Now we have the basic development setup and process in place. We can build the code into executable firmware images. We can flash these images to a connected development board, and we can monitor the console to look at debug logs and messages generated by the firmware. 

Let's now build a simple power outlet with ESP32.

%----------------------------------------------------------------------------------------
%	CHAPTER 3
%----------------------------------------------------------------------------------------

\chapterimage{band1.jpg} % Chapter heading image

\chapter{The Outlet} \label{the-outlet}

In this Chapter we will create a basic power outlet using the driver APIs of the ESP32. The power outlet will do the following:
\begin{itemize}
    \item Provide a button that the user can press
    \item Toggle an output GPIO on every button press
\end{itemize}
For the scope of this chapter, we won't worry about 'connectivity' of this power outlet. That will follow in subsequent chapters. Here we will only focus in implementing the outlet functionality. You may refer to the \textit{2outlet/} directory of esp-jumpstart for looking at this code. 

The code for the driver has been neatly isolated in the file \textit{app\_driver.c}. This way, later whenever you have to modify this application to adapt to your product, you could simply change the contents of this file to talk to your peripheral.

\section{The Push Button}\index{The Push Button}
Let's first create a push-button. The Devkit-C development board has a button called 'boot' which is connected to GPIO 0. We will configure this button to be used to toggle the outlet's state.

\ksnotebox{If you are developing with a different development board, please use the appropriate GPIO number for the button that this board may have.}

\subsection{The Code}\index{The Code}\label{sec:push_button}
The code for enabling this is shown as below:
\begin{minted}{c}
#include <iot_button.h>

button_handle_t btn_handle=iot_button_create(BUTTON_GPIO,
                                BUTTON_ACTIVE_LEVEL);
iot_button_set_evt_cb(btn_handle, BUTTON_CB_RELEASE,
                            push_btn_cb, "RELEASE");

\end{minted}

We use the \textit{iot\_button} module for implementing the button. 
First off we create the iot\_button object. We specify the GPIO number and the active level of the GPIO to detect the button press. In the case of DevKit-C the \textit{BUTTON\_GPIO} is set to GPIO 0. 

Then we register an event callback for the button, whenever the button is \textit{released} the \textit{\textbf{push\_btn\_cb}} function will be called. This function is called in the esp-timer thread's context. So do make sure that the default stack configured for the esp-timer thread is sufficient for your callback function.

The \textit{push\_btn\_cb} code then is simply as shown below:
\begin{minted}{c}
static void push_btn_cb(void* arg)
{
    static uint64_t previous;
    uint64_t current = xTaskGetTickCount();
    if ((current - previous) > DEBOUNCE_TIME) {
        previous = current;
        app_driver_toggle_state();
    }
}
\end{minted}

The \textit{xTaskGetTickCount()} is a FreeRTOS function that provides the current tick counts. In the callback function, we make sure that the button press doesn't accidentally generate multiple events in a short duration of time. This is generally not what the end-user wants. (In the current case, we absorb all events generated within a 300 millisecond span, and call it a single event.)
Finally, we call the function \textit{app\_driver\_toggle\_state()} which is responsible for toggling the output on or off.

\section{The Output}\index{The Output}
Now we will configure a GPIO to act as the output of the power outlet. We will assert this GPIO on or off which would ideally trigger a relay to switch the output on or off.

\subsection{The Code}\index{The Code}\label{sec:relay}
First off we initialize the GPIO with the correct configuration as shown below:

\begin{minted}{c}
gpio_config_t io_conf;
io_conf.mode = GPIO_MODE_OUTPUT;
io_conf.pull_up_en = 1;
io_conf.pin_bit_mask = ((uint64_t)1 << OUTPUT_GPIO);

/* Configure the GPIO */
gpio_config(&io_conf);

\end{minted}

In this example, we have chosen GPIO 27 to act as the output. We initialize the \textit{gpio\_config\_t} structure with the settings to set this as a GPIO output with internal pull-up enabled.

\begin{minted}{c}
/* Assert GPIO */
gpio_set_level(OUTPUT_GPIO, target);

\end{minted}

Finally, the state of the GPIO is set using the \textit{gpio\_set\_level()} call.

\section{Progress so far}\index{Progress so far}
With this, now we have a power outlet functionality enabled. Once you build and flash this firmware, every time the user presses the push-button the output from the ESP32 toggles on and off. As of now, this is not a connected outlet though. 

As our next step, let's add Wi-Fi connectivity to this firmware.

%----------------------------------------------------------------------------------------
%	CHAPTER 4
%----------------------------------------------------------------------------------------

\chapterimage{band1.jpg} % Chapter heading image

\chapter{Connecting to Wi-Fi}
Let's now get this power outlet on a Wi-Fi network. In this Chapter we will connect to a hard-coded Wi-Fi network that is embedded within the device's firmware executable image. You may refer to the \textit{3wifi\_connection/} directory of esp-jumpstart for looking at this code.

Wi-Fi is a protocol that can generate asynchronous events like connectivity lost, connection established, DHCP Address received etc. For this, we register a handler with the Wi-Fi and network subsystem. This handler will get called whenever either of these asynchronous events occurs.

\section{The Code}\index{The Code}
\begin{minted}{c}
#include <esp_wifi.h>
#include <esp_event_loop.h>

tcpip_adapter_init();
esp_event_loop_init(event_handler, NULL);

wifi_init_config_t cfg = WIFI_INIT_CONFIG_DEFAULT();
esp_wifi_init(&cfg);
esp_wifi_set_mode(WIFI_MODE_STA);

wifi_config_t wifi_config = {
    .sta = {
        .ssid = EXAMPLE_ESP_WIFI_SSID,
        .password = EXAMPLE_ESP_WIFI_PASS,
    },
 };
esp_wifi_set_config(ESP_IF_WIFI_STA, &wifi_config);
esp_wifi_start();

\end{minted}

In the above code:
\begin{itemize}
    \item We initialize the TCP/IP stack with the \textit{tcpip\_adapter\_init()} call
    \item Similarly, the Wi-Fi subsystem and its station interface is initialized with the calls to \textit{esp\_wifi\_init()} and \textit{esp\_wifi\_set\_mode()}
    \item Finally, the hard-coded SSID and passphrase configuration of the target Wi-Fi network are configured and we start the station using a call to \textit{esp\_wifi\_start()}
\end{itemize}

The call to \textit{esp\_event\_loop\_init()} is important. The event loop collects events from the TCP/IP Stack and the Wi-Fi subsystem. It delivers these events to the callback that is registered through the first parameter.

\ksnotebox{The callback handler is executed from the event loop task. Care should be taken to ensure that this callback's execution doesn't overflow the event loop's stack. If you need a deeper stack in your callback, you can increase the event loop's stack size by changing the SDK configuration.}

The asynchronous event handler that is registered with the event loop can be implemented as:
\begin{minted}{c}
esp_err_t event_handler(void *ctx, system_event_t *event)
{
    switch(event->event_id) {
    case SYSTEM_EVENT_STA_START:
        esp_wifi_connect();
        break;
    case SYSTEM_EVENT_STA_GOT_IP:
        ESP_LOGI(TAG, "Connected with IP Address:%s",  
             ip4addr_ntoa(&event->event_info.got_ip.ip_info.ip));
        break;
    case SYSTEM_EVENT_STA_DISCONNECTED:
        esp_wifi_connect();
        break;
    return ESP_OK;
}
\end{minted}

The event handler current handles 3 events. When it receives an event \textit{SYSTEM\_EVENT\_STA\_START}, it asks the station interface to connect using the \textit{esp\_wifi\_connect()} call. The same action is taken even when we receive a Wi-Fi disconnect event.

The event \textit{SYSTEM\_EVENT\_STA\_GOT\_IP} is received when a DHCP IP address is obtained by ESP32. In this particular case, we  only print the IP address on the console.

\section{Progress so far}\index{Progress so far}
You can now modify the application to enter your Wi-Fi network's SSID and the passphrase. When you compile and flash this code on your development board, the ESP32 should connect to your Wi-Fi network and print the IP address on the console. The outlet's functionality of toggling the GPIO on pressing the push-button is, of course, also retained.

One problem with this approach is that the Wi-Fi settings are hard-coded into the firmware image. While this is ok for a hobby project, a product will require the end-user to dynamically configure this device with their settings. This is what we will look at in the next chapter.

%----------------------------------------------------------------------------------------
%	CHAPTER 5
%----------------------------------------------------------------------------------------

\chapterimage{band1.jpg} % Chapter heading image

\chapter{Network Configuration}

In this step we will build a firmware such that the end-user can configure her Wi-Fi network's credentials into the device at run-time. Since a user's network credentials will be stored persistently on the device, we will also provide a \textit{Reset to Factory} action where a user's configurations can be erased from the device.
You may refer to the \textit{4network\_config/} directory of esp-jumpstart for looking at this code.

In the previous example, we had hard-coded the Wi-Fi credentials into the firmware. This obviously doesn't work for a end-user product.

\section{Overview}\index{Overview}
As can be seen in this figure, in the network configuration stage, the end-user typically uses her smart-phone to \textit{securely} configure her Wi-Fi credentials into your device. Once the devices acquires these credentials, it can then connect to her home Wi-Fi network. 

\begin{figure}
    \centering
    \includegraphics[scale=0.4]{Pictures/network_config.png}
    \caption{Network Configuration Process}
    \label{fig:network_config}
\end{figure}

There can be multiple channels through which your device can receive the Wi-Fi credentials. ESP32 supports the following mechanisms:

\begin{itemize}
    \item SoftAP 
    \item Bluetooth Low Energy (BLE)
    \item Smart-Config
\end{itemize}

Each of these have their own pros and cons. There is no single correct way of doing this, some developers may pick one way, and some the other, depending upon what you value more.

\subsection{SoftAP}\index{SoftAP}
In the SoftAP mechanism your outlet will launch its own temporary Wi-Fi Access Point. The user can then connect their smart-phones to this temporary Wi-Fi network. And then use this connection to transfer the Home Wi-Fi's credentials to the outlet. Many connected devices in the market today, like the Google Chromecast, Amazon's Echo use this kind of mechanism. In this network configuration workflow, the user has to 
\begin{itemize}
    \item switch their phone's Wi-Fi network to your outlet's temporary Wi-Fi network
    \item launch your phone application
    \item enter her home Wi-Fi credentials that will be then transferred to the outlet over the SoftAP connection
\end{itemize}
From a user experience perspective, the first step of this requires the user to change their phone's Wi-Fi network. This may be confusing to some users. Additionally, changing the Wi-Fi network programatically through the phone application may not always be possible (iOS and some variants of Android don't allow application to this). The advantage of this method though is that it is very reliable (SoftAP being just Wi-Fi is an established mechanism), and doesn't require a lot of additional code (since it's all over Wi-Fi).

\subsubsection{Apple's WAC}\index{Apple's WAC}
Apple's Wireless Accessory Configuration (WAC) protocol also uses the SoftAP mechanism for transferring credentials. But in this case, iOS itself manages the switching of Wi-Fi networks between the device's network and the user's home network. Since this support is embedded within iOS itself, this offers a much smoother user experience.
One point to note though is that Apple's WAC configuration mechanism mandates that your device, the outlet, should have Apple's security co-processor (a chip) embedded on your board. So the simplified user experience comes at the cost of having another chipset embedded within your board design.

\subsection{BLE}\index{BLE}

In the Bluetooth Low Energy (BLE) method, your outlet will be doing a BLE advertisement. Phones in the vicinity can see this advertisement, and ask the user to do a BLE connection with your device. Then this network is used to transfer the credentials to the outlet.
In this network configuration workflow, the user doesn't have to do the hard task of switching between Wi-Fi networks. Additionally, both iOS and Android allow phone application to scan for BLE devices in the vicinity and also connect to them through the app. This means a much smoother end-user experience.

One side-effect, though, of using the BLE based network configuration is that it also pulls in the code for Bluetooth. This means your flash requirement may be affected since your firmware size will increase. During the network configuration mode, BLE will also consume memory until the network configuration is complete.

\subsection{Smart-Config}\index{Smart-Config}
XXX

\section{Demo}\index{Demo}
Before getting into the details of the network configuration workflow, let us get a feel for how an end-user will configure the network using the provided application.
You may refer to the \textit{4network\_config/} directory of esp-jumpstart for trying this out.

\begin{itemize}
    \item Go to the \textit{4network\_config} application.
    \item Build, flash and load the application.
    \item By default, the firmware is launched in BLE mode.
    \item Install the companion phone application for network configuration from this location: \url{https://github.com/espressif/esp-idf-provisioning-android/releases}. Please install the latest app with \textbf{sec1-ble} as part of its name.
    \item Launch the application and follow the wizard as shown in the images below.
\end{itemize}

\ksnotebox{As of now, only Android users can try the application out. For iOS users, we are working on enabling an application with \textit{TestFlight} soon.}

XXX add network configuration images

\begin{itemize}
    \item If all goes well, your device would be connected to your Home Wi-Fi network.
    \item If you now reset the device, it will not enter the network-configuration mode. Instead it will go and connect to the Wi-Fi network that is configured.
\end{itemize}

 
\section{Unified Provisioning}\index{Unified Provisioning}

Espressif provides a \textbf{Unified Provisioning} module for assisting you with your network configuration. When this module is invoked from your firmware executable, the module takes care of managing all the state transitions (like starting/stopping the softAP/BLE interface, exchanging the credentials securely, storing them for subsequent use etc).

\begin{itemize}

\item Extensible Protocol: The protocol is completely flexible and it offers the ability for the developers to send custom configuration in the provisioning process. The data representation too is left to the application to decide.
\item Transport Flexibility: The protocol can work on Wi-Fi (SoftAP + HTTP server) or on BLE as a transport protocol. The framework provides an ability to add support for any other transport easily as long as command-response behaviour can be supported on the transport.
\item Security Scheme Flexibility: It’s understood that each use-case may require different security scheme to secure the data that is exchanged in the provisioning process. Some applications may work with SoftAP that’s WPA2 protected or BLE with “just-works” security. Or the applications may consider the transport to be insecure and may want application level security. The unified provisioning framework allows application to choose the security as deemed suitable.
\item Compact Data Representation: The protocol uses Google Protobufs as a data representation for session setup and Wi-Fi provisioning. They provide a compact data representation and ability to parse the data in multiple programming languages in native format. Please note that this data representation is not forced on application specific data and the developers may choose the representation of their choice.

\end{itemize}

\ksnotebox{More details about Unified provisioning are available at: \url{https://docs.espressif.com/projects/esp-idf/en/latest/api-reference/provisioning/provisioning.html}}

The following components are offered:
\begin{itemize}
    \item \textbf{Unified Provisioning Specification:} A specification to \textit{securely} transfer Wi-Fi credentials to the device, independent of the transport (SoftAP, BLE)
    \item \textbf{IDF Components:} Software modules that implement this specification in the device firmware, available through ESP-IDF
    \item \textbf{Phone Libraries:} Reference implementations on iOS and Android are available that can be directly incorporated into your existing phone applications
    \item \textbf{Reference Phone Applications:} Fully functional Phone applications on Android (\url{https://github.com/espressif/esp-idf-provisioning-android}) and iOS (\url{https://github.com/espressif/esp-idf-provisioning-ios}) are available for testing during your development, or for skinning with your brand's elements.
\end{itemize}

\subsection{The Code}\index{The Code}\label{sec:unified_prov}
The code for invoking the unified provisioning through your firmware is shown below:
\begin{minted}{c}

if (conn_mgr_prov_is_provisioned(&provisioned) != ESP_OK) {
    return;
}

if (! provisioned) {
    /* Starting unified provisioning */
    conn_mgr_prov_start_provisioning(prov_type,
               security, pop, service_name, service_key);
} else {
    /* Start the station */
    wifi_init_sta();
}
\end{minted}

The \textit{conn\_mgr\_prov} component provides a wrapper over the unified provisioning interface. Some notes about the code above:
\begin{itemize}
    \item The \textit{conn\_mgr\_prov\_is\_provisionined()} API checks whether Wi-Fi network credentials have already been configured or not. These are typically stored in a flash partition called the \textit{NVS}. More about NVS later in this Chapter.
    \item If no Wi-Fi network credentials are available, the firmware launches the unified provisioning using the call \textit{conn\_mgr\_prov\_start\_provisioning()}. This API will take care of everything, specifically:
    \begin{enumerate}
        \item It will start the SoftAP or BLE transport as configured
        \item It will enable the necessary advertisements using the Wi-Fi or BLE standards
        \item It will \textit{securely} accept any network credentials from a phone application
        \item It will store these credentials, for future use, in the NVS
        \item Finally, it will deinitialise any components (SoftAP, BLE, HTTP Server etc) that were required by the unified provisioning mechanism. This ensures that this point onward there is almost no memory overhead from the unified provisioning module.
    \end{enumerate}
    \item If a Wi-Fi network configuration was found in NVS, we directly start the Wi-Fi station interface using \textit{wifi\_init\_sta()}.
\end{itemize}

These steps ensure that the firmware launches the unified provisioning module when no configuration is found, and if a configuration is available, then starts the Wi-Fi station interface.

Additionally, the unified provisioning module also needs to know the state transitions of the Wi-Fi interface. Hence an additional call needs to be made from the event handler for taking care of this:
\begin{minted}{c}
esp_err_t event_handler(void *ctx, system_event_t *event)
{
     conn_mgr_prov_event_handler(ctx, event);
   
     switch(event->event_id) {
     case SYSTEM_EVENT_STA_START:
...
...
...
\end{minted}

\subsubsection{Configurable Options}\index{Configurable Options}
In the code above, we have used the following call for invoking the unified provisioning interface:
\begin{minted}{c}
    /* Starting unified provisioning */
    conn_mgr_prov_start_provisioning(prov_type,
               security, pop, service_name, service_key);
\end{minted}

Let us now look at the parameters, or the configuration options of this API:
\begin{enumerate}
    \item \textbf{Transport:} The developer can choose which transport mechanism will be used for the network configuration. The options available are SoftAP or BLE. 
    \begin{itemize}
        \item The module is written in such a manner that, based on the developer's selection, only the relevant software libraries will get pulled into the final executable image. 
        \item The unified provisioning module will also manage the state transitions, and other services, that are required for the network configuration to take place
    \end{itemize}
    \item \textbf{Service Name:} When the user launches the network configuration app, the user will be presented with a list of unconfigured devices, in her vicinity. The service name is this name that will be visible to the user. You may choose a name that identifies your device conveniently (abc-thermostat). It is common practice to have some element in the service name that is unique or random. This helps in scenarios when there could be multiple unconfigured devices that the user is configuring at the same time.
    \item \textbf{Proof of Possession:} When a user brings in a new smart device, the device launches its provisioning network (BLE, SoftAP) for configuration.  How do you make sure that only the owner of the device configures the device and not their neighbours? This configurable option is for that. Please read the following subsection for more details about this option.
    \item \textbf{Security:} The unified provisioning module currently supports two security methods for transferring the credentials: \textit{security0} and \textit{security1}. Security0 uses no security for exchanging the credentials. This is primarily used for development purposes. Security1 uses elliptic curve, \textit{curve25519} crypto for key exchange, followed by \textit{AES-CTR} encryption for data exchanged on the channel.
\end{enumerate}

\subsubsection{Proof of Possession}\index{Proof of Possession}

When a user brings in a new smart device, the device launches its provisioning network (BLE, SoftAP) for configuration.  How do you make sure that only the owner of the device configures the device and not their neighbours?

Some products expect the user configuring the device to provide a proof that they really own (or posses) the device that they are configuring. The proof of possession can be provided by taking some physical action on the device, or by entering some unique random key that is pasted on the device's packaging box, or by displaying on a screen, if the device is equipped with one.

At manufacturing, every device can be programmed with a unique random key. This key could then be provided to the unified provisioning module as a proof of possession option. When the user configures the device using the phone application, the phone application transfers the proof of possession to the device. The unified provisioning module then validates that the proof of possession matches and then confirms the configuration.

\subsection{Additional Details}\index{Additional Details}

More details about Unified provisioning are available at: \url{https://docs.espressif.com/projects/esp-idf/en/latest/api-reference/provisioning/provisioning.html}

\section{NVS: Persistent key-value store}\index{NVS: Persistent key-value store}
In the Unified Provisioning section above, we mentioned in passing that the Wi-Fi credentials are stored in the NVS. The NVS is a software component that maintains a persistent storage of key-value pairs. Since the storage is persistent this information is available even across reboots and power shutdowns. The NVS uses a dedicated section of the flash to store this information.

The NVS is designed in such a manner so as to be resilient to metadata corruption across power loss events. It also takes care of wear-levelling of the flash by distributing the writes throughout the NVS partition.

Application developers can also use the NVS to store any additional data that you wish to maintain as part of your application firmware. Data types like integers, NULL-terminated strings and binary blobs can be stored in the NVS. This can be used to maintain any user configurations for your product. Simple APIs like the following can be used to read and write values to the NVS.

\begin{minted}{c}
  /* Store the 'chosen_value' variable to NVS */
  nvs_set_u32(nvs_handle, "my_key", chosen_value);

  /* Read the 'chosen_value' variable from NVS */
  nvs_get_u32(nvs_handle, "my_key", &chosen_value);
\end{minted}


\subsection{Additional Details}\index{Additional Details}

More details about NVS are available at: \url{https://docs.espressif.com/projects/esp-idf/en/latest/api-reference/storage/nvs_flash.html}

\section{Reset to Factory}\index{Reset to Factory}
Another common behaviour that is expected of products is \textit{Reset to Factory Settings}. Once the user configuration is stored into the NVS as discussed above, reset to factory behaviour can be achieved by simply erasing the NVS partition.

Generally, this action is triggered by long-pressing a button available on the product. This can easily be configured using the \textit{iot\_button\_*()} functions

\subsection{The Code}\index{The Code}\label{sec:reset_to_factory}
In the \textit{4network\_config/} application, we use a long-press action of the same toggle push-button to configure the reset to factory behaviour.

\begin{minted}{c}
/* Register 3 second press callback */  
iot_button_add_on_press_cb(btn_handle, 3, button_press_3sec_cb, NULL);
\end{minted}

This function makes the configuration such that the \textit{button\_press\_3sec\_cb()} function gets calls whenever the button associated with the \textit{btn\_handle} is pressed and released for longer than 3 seconds. Remember we had initialised the \textit{btn\_handle} in our Chapter \ref{the-outlet}

Then callback function can then be written as follows:
\begin{minted}{c}
static void button_press_3sec_cb(void *arg)
{
    nvs_flash_erase();
    esp_restart();
}
\end{minted}

This code basically erases all the contents of the NVS, and then triggers a restart. Since the NVS is now wiped, the next time the device boots-up it will go back into the unconfigured mode. 

If you have loaded and configured the device with the \textit{4network\_config/} application, you can see this in action and by pressing the toggle button for more than 3 seconds and then releasing it.

\section{Progress so far}\index{Progress so far}
Now we have a smart outlet that the user can configure, through a phone app, to their home Wi-Fi network. Once configured, the outlet will keep connecting to this configured network. We also have the ability to erase these settings on a long-press of a push-button.

As of now, the outlet functionality and the connectivity functionality are separate. As our next step, let's control and monitor the state of the outlet (on/off) remotely.

%----------------------------------------------------------------------------------------
%	CHAPTER 6
%----------------------------------------------------------------------------------------

\chapterimage{band1.jpg} % Chapter heading image

\chapter{Remote Control (Cloud)}

The potential for smart connected devices can be explore when the connectivity is used to control or monitor the device remotely, or through integration with other services. This is where the cloud communication comes into picture. In this Chapter, we will get the connected to a cloud platform, and enable remote control and monitoring of the device.

Typically, this is achieved through either of the scenarios as shown in \ref{fig:cloud_connectivity}.

\begin{figure}
    \centering
    \includegraphics[scale=0.4]{Pictures/CloudConnectivity.png}
    \caption{Value of Cloud Connectivity}
    \label{fig:cloud_connectivity}
\end{figure}

In most cases, once a device is connected to the cloud, the Device Cloud platforms expose the device control and monitoring through a RESTful web API. Authenticated clients, like smartphone apps can use these APIs to access the device remotely.

Additionally, integration with other clouds also helps in realising valuable use cases. For example, the device can be linked with a weather information system to automatically tune itself, or it can be linked to voice-assistant cloud interfaces (like Alexa or Google Voice Assistant) to expose control through voice.

\section{Security First}\index{Security First}
Before we get into the details about cloud connectivity, a few important words about security. 

Connecting with any remote cloud infrastructure must always happen using TLS (Transport Layer Security). It is a standard and it takes care of ensuring that communication stays secure. This is a transport layer protocol. Any higher-level protocols like HTTP, or MQTT can use TLS as the underlying transport. All reputable cloud vendors provide device services over TLS.

\subsection{CA Certificates}\index{CA Certificates}
One aspect of TLS is server validation using CA certificates. The TLS layer uses the CA certificate to validate that you are really talking to the server that you are supposed to talk to. For this validation to happen, your device must be pre-programmed with one or more valid and trusted CA certificate. The TLS layer will use these as trusted certificates and then validate the server based on these trusted certificates.

\subsection{Additional Details}\index{Additional Details}

More details about TLS and certificates is available at: \url{https://medium.com/the-esp-journal/esp32-tls-transport-layer-security-and-iot-devices-3ac93511f6d8}


\section{AWS IoT}\index{AWS IoT}
In this section we will connect the device to Amazon's AWS IoT cloud. 

\subsection{Quick Setup}\index{Quick Setup}
XXX Specify quick setup instructions

\subsection{Demo}\index{Demo}
By the end of the previous sub-section, you should have the following items ready to get your device to start talking with AWS IoT:
\begin{enumerate}
    \item A Device Private Key (a file)
    \item A Device Certificate (a file)
    \item A Thing Name (a string)
    \item A CA Certificate for the AWS-IoT service's domain name (a file)
\end{enumerate}

Before getting into the details of the code, let us actually try to use the remote control for our device.
You may refer to the \textit{4cloud/} directory of esp-jumpstart for trying this out.

To setup your AWS IoT example, 
\begin{enumerate}
    \item Go to the \textit{5cloud/} application
    \item Copy the files (overwriting any previous files) as mentioned below:
    \begin{itemize}
        \item The AWS CA Certificate to \textbf{5cloud/certs/aws-root-ca-.pem}
        \item The Device Private Key to \textbf{5cloud/certs/private.pem.key}
        \item The Device Certificate to \textbf{5cloud/certs/certificate.pem.crt}
    \end{itemize}
    \item Modify the thing name \textbf{EXAMPLE\_THING\_NAME} in the file \textit{5cloud/main/cloud.c}
    \item Build, flash and load the firmware on your device
\end{enumerate}

The firmware is so written that it will now connect to the AWS IoT cloud platform and will notify the cloud of any state changes. The firmware will also fetch any updates to the state from the cloud and apply them locally. 

\subsection{Remote Control}\index{Remote control}
With reference to the Section \ref{fig:cloud_connectivity}, the AWS IoT exposes a RESTful web API for all devices that connect to it. Phone applications can interact with this Web API to control and monitor the device. We will use cURL, a command-line utility that can be used to simulate this phone app. 

Using curl, we can then read the current state of the device by executing the following command on your Linux/Windows/Mac console:
\begin{minted}{console}

curl --tlsv1.2 --cert /work/certificate.pem.crt \
       --key /work/private.pem.key   \
       https://aln7lww42a72l-ats.iot.us-east-2.amazonaws.com:8443/things/my_device_name/shadow \ 
       | python -mjson.tool

\end{minted}

AWS expects that access to a device state is only granted to entities that are authorised to do so. Hence in the command above, we use the \textit{certificate.pem.crt} and \textit{private.pem.key}, which are the same files that we have configured to be in the firmware. This ensures that we can access the device's state.

In the command above, this reads the state from the device \textbf{my\_device\_name}. Don't forget to replace this with the name of your thing.

The device state can be modified as:
\begin{minted}{console}

curl -d '{"state":{"desired":{"output":false}}}' \ 
     --tlsv1.2 --cert /work/certificate.pem.crt \ 
     --key /work/private.pem.key \ 
     https://aln7lww42a72l-ats.iot.us-east-2.amazonaws.com:8443/things/my_device_name/shadow \ 
     | python -mjson.tool
\end{minted}

This cURL command will generate an HTTP POST operation, and sends the JSON data, as shown above, as the post's body. This JSON data instructs AWS IoT to update the state of the device to false.

You can observe the corresponding change of state on the device whenever you change the state from cURL to true or false.

So that's how remote control is achieved. Let's now quickly talk about the code.

\subsection{The Code}\index{The Code}
All the code for the cloud communication has been consolidated in the \textit{cloud.c} file. The structure of this file is similar to what the standard AWS IoT SDK expects. 

The file uses our output driver's APIs, \textit{app\_driver\_get\_state()} and \textit{app\_driver\_toggle\_state()}, to fetch and modify the device state respectively.


\section{Embedding Files in the Firmware}\index{Embedding Files in the Firmware}
The device key, certificate and the CA certificate are files that the firmware should use to communicate with AWS IoT. The question is how do you make the entire contents of these files be part of your firmware image, and how do you access them within your firmware?

ESP-IDF provides a great mechanism for enabling this. The \textit{component.mk} file can be used to inform the build system that the contents of certain files should be embedded within firmware image. This can be enabled by adding the following line into your application's \textit{component.mk} file.


%----------------------------------------------------------------------------------------
%	CHAPTER 7
%----------------------------------------------------------------------------------------

\chapterimage{band1.jpg} % Chapter heading image

\chapter{Firmware Upgrades}

\section{Flash Partitions}\index{Flash Partitions}

%----------------------------------------------------------------------------------------
%	CHAPTER 8
%----------------------------------------------------------------------------------------

\chapterimage{band1.jpg} % Chapter heading image

\chapter{Manufacturing}


%----------------------------------------------------------------------------------------
%	CHAPTER 9
%----------------------------------------------------------------------------------------

\chapterimage{band1.jpg} % Chapter heading image

\chapter{For the restless}
XXX Consider moving this to the start of the book
For the restless, here are the quick steps and pointers to additional information.
\begin{enumerate}
    \item Get Jumpstart, IDF repositories and setup your host
    \item Read on the usage of the GPIO driver and then create a push-button (\ref{sec:push_button}) action that drives an output GPIO (\ref{sec:relay})
    \item Use the unified provisioning API to get the devices on the user's Wi-Fi network (\ref{sec:unified_prov}). This will store the Wi-Fi name and password in NVS
    \item Use the reference phone-app (iOS/Android) libraries or apps for building your phone applications
    \item Implement the \textit{Reset to Factory Settings} by erasing the Wi-Fi credentials stored in NVS (\ref{sec:reset_to_factory})
\end{enumerate}

\clearpage
\vspace*{\fill} 
\centering \Huge {\textit{Happy Productising!}}
\vspace*{\fill} 

\end{document}
